\Transcb{yellow}{blue}{Formalism of a Bayesian rate analysis}
\onecolumn
\begin{center}
{\red The problem to be addressed} 
\end{center}
\begin{itemize}
\item Cosmic rays impinge on the atmosphere of the Earth, producing a constant rate
      of atmospheric high-energy neutrinos, homogeneously distributed over the celestial sphere.
      We call these neutrinos (atmospheric) background.
\item Astrophysical sources may yield an additional, constant high-energy neutrino rate at specific locations.
      We call these neutrinos (cosmic) signal.
\item[$\ast$] {\blue The essential questions :}
\begin{itemize}
\item Can we identify a possible signal by measurements ?
\item Can we determine (a limit on) the source strength ?
\item What is our degree of belief in the presence of a source ?
\end{itemize}
\item To investigate we study a certain patch on the sky over a time interval $\Delta t$.
\item[] This will result in observing $n$ neutrinos.
\item[$\ast$] All of the above is called our {\blue prior information $I$}.
\item Note : {\red The following reasoning works also for stacked observations.}
\end{itemize}

\Tr
\begin{itemize}
\item Based on our prior information we know that the pdf for the number of\\
      observed neutrinos $n$ is the Poisson distribution with constant rate $r$
\item[] \begin{center} {\blue $\displaystyle p(n|rI)=\frac{(r\Delta t)^{n} e^{-r\Delta t}}{n!}$} \end{center}
\item But : We actually want to determine the rate $r \rightarrow p(r|nI)$
\end{itemize}
%
\begin{center}
{\red Bayesian Logical Inference} 
\end{center}
\begin{itemize}
\item Consider two {\red propositions $H$ and $D$} and some {\red prior information $I$}
\item[] {\red Product rule : $p(HD|I)=p(H|I)p(D|HI)=p(D|I)p(H|DI)$}
\item From the product rule we have : $p(H|I)p(D|HI)=p(D|I)p(H|DI)$
\item[] which can be written as :
\end{itemize}
%
\begin{center}
{\red \shabox{$\displaystyle p(H|DI)=p(H|I)\,\frac{p(D|HI)}{p(D|I)}$}\\[5mm]
 (Theorem of Bayes)}
\end{center}

\Tr
\begin{itemize}
\item The Bayes theorem directly yields {\blue $\displaystyle p(r|nI)=p(r|I)\,\frac{p(n|rI)}{p(n|I)}$}
\item[] $p(r|I)$ is some prior pdf for the rate
\item[] $p(n|rI)$ is the Poisson pdf from before
\item[] $p(n|I)$ is some normalisation which can be determined as follows :
\item[] $\int p(r|nI)\,\d r = 1 \rightarrow {\blue p(n|I)=\int p(r|I)\,p(n|rI)\,\d r}$
\item[$\ast$] But : {\red The rate $r$ consists of independent signal and background $\rightarrow r=r_{s}+r_{b}$}
\item[] So we get : {\blue $\displaystyle p(r_{s}r_{b}|nI)=p(r_{s}r_{b}|I)\,\frac{p(n|r_{s}r_{b}\,I)}{p(n|I)}$}
\item[] where : {\blue $\displaystyle p(n|r_{s}r_{b}\,I)=\frac{([r_{s}+r_{b}]\Delta t)^{n} e^{-[r_{s}+r_{b}]\Delta t}}{n!}$}
        (Poisson for rate $r=[r_{s}+r_{b}]$)
\item[] Product rule : $p(r_{s}r_{b}|I)=p(r_{b}|I)p(r_{s}|r_{b}\,I)=p(r_{b}|I)p(r_{s}|I)$
\begin{itemize}
\item[] $p(r_{b}|I)$ is some prior pdf for the bkg rate (e.g. $p(r_{b}|n_{b}\,I)$ from off-source) 
\item[] $p(r_{s}|I)$ is some prior pdf for the signal rate (e.g. based on previous limits) 
\end{itemize}
\item Similar as above : {\blue $p(n|I)=\int \int p(r_{b}|I)\,p(r_{s}|I)\,p(n|r_{s}r_{b}\,I)\,\d r_{b}\,\d r_{s}$}
\end{itemize}

\Tr
\begin{itemize}
\item Conclusion :
\item[] {\blue Given some priors $p(r_{b}|I)$ and $p(r_{s}|I)$ we can determine $p(r_{s}r_{b}|nI)$}
\item[] But : {\red We want to determine the pure signal rate $p(r_{s}|nI)$}
\item[$\ast$] Thanks to the Bayesian logic : {\blue Marginalisation}
\item[] Without loss of statistical information : {\blue $p(r_{s}|nI)=\int p(r_{s}r_{b}|nI)\,\d r_{b}$}
\item[] \colorbox{yellow}{We can determine the full posterior signal rate pdf $p(r_{s}|nI)$ from data alone}
\item {\blue Going to credible regions, upper limits and all that}
\item[$\ast$] x\% credibility region $[r_{min},r_{max}]$ for $r_{s}$ : $\displaystyle \int_{r_{min}}^{r_{max}} p(r_{s}|nI)\,\d r_{s}=$ x\%
\item[] with $r_{min} < \hat{r}_{s} < r_{max}$ and $p(r_{min}|nI)=p(r_{max}|nI)$ 
\item[] $\rightarrow r_{min}$ and $r_{max}$ form the x\% credibility region of the signal rate $r_{s}$
\item[$\ast$] x\% upper limit $r_{max}$ for $r_{s}$ : $\displaystyle \int_{0}^{r_{max}} p(r_{s}|nI)\,\d r_{s}=$ x\%
\item[] $\rightarrow r_{max}$ is the x\% credible upper limit for the signal rate $r_{s}$
\end{itemize}

\Tr
\begin{itemize}
\item {\blue Decision between credibility region or upper limit}
\item[] Degree of belief in signal presence via hypothesis testing c.q. $\psi$ value
\item[] (NvE, Astroparticle Physics 28 (2008) 540, arXiv:astro-ph/0702029)
\item[] See also my lectures on "Logical Data Analaysis" 
\item {\blue Going from an observed signal rate $r_{s}$ to an incoming flux value $F_{s}$}
\item[] Just plug in the {\red Effective Area} ($A_{eff}$)
\item[] $A_{eff} \equiv$ observed event rate~/~incoming flux $\quad$ (from simulations)
\item[] which yields : $\displaystyle F_{s}=\frac{r_{s}}{A_{eff}}
        \rightarrow {\blue p(F_{s}|nI)=\frac{p(r_{s}|nI)}{A_{eff}}}$
\item[] \colorbox{yellow}{We get the full posterior source flux pdf $p(F_{s}|nI)$ directly from $p(r_{s}|nI)$}
\item {\blue Derivation of the (distance independent) source intensity $I_{s}$}
\item[] In case the solid angle $\Omega_{s}$ of the source can be determined :
        $\displaystyle I_{s}=\frac{F_{s}}{\Omega_{s}}$ 
\end{itemize}
